%!TEX root = ../../report.tex


\subsection{Conclusions}

Refactoring tools for dynamic languages are still far away from the capabilities offered by the refactoring tools for static languages and in average have less refactoring operations than the refactoring tools for static languages.

Dynamic languages have the problem of having less information available in compile time and that might explain the different capabilities of the refactoring tools for dynamic languages when compared to refactoring tools for static languages. 
%The dynamic languages are also very different from each other, whereas the static languages such as Java or C\# are similar.

The dynamic langauges even having few refactoring operations, the available ones are very different from one language to another. However, the rename, which is the most used refactoring operations, is available in every dynamic refactoring tool.

%\begin{table}[h]
%\label{tab-Comparing-All}
%\begin{tabular}{l|c|c|c|c}
%              & \multicolumn{1}{l|}{Dynamic?} & Refactoring Coverege & Interoperability & Major drawback				 		\\ \hline
%Griswold      & Yes                           & Medium               &                  &									    \\ \hline
%SmallTalk     & Yes                           & Medium               & No             	& Dependent of Unit tests \& maybe dead \\ \hline
%Javascript    & Yes                           & Very Low             &                  & Few Refactoring Operations 		    \\ \hline
%Bicycle       & Yes                           & Medium               & Yes              & Did not improve since 2004 			\\ \hline
%Rope          & Yes                           & Medium               & Yes              &                       			    \\ \hline
%Eclipse       & No                            & High                 & No               &                            			\\ \hline
%Visual Studio & No                            & Medium               & No               &                            			\\ \hline
%IntelliJ      & No                            & High                 & No               &                            			\\ \hline
%NetBeans      & No                            & Low                  & No               &                            			\\ \hline
%JBuilder      & No                            & Medium               & No               &                            			\\ \hline
%\end{tabular}
%\caption{Comparassion between Refactoring tools}
%\end{table}