%!TEX root = ../../report.tex


\subsection{Conclusions}

Refactoring tools for dynamic languages are still far away from the capabilities offered by the refactoring tools for static languages and in average have less refactoring operations than the refactoring tools for static languages.

One reason to the different capabilities is the problems that refactoring tools have to deal with than the refactoring tools for static languages, beside having the problem of having less information in compile time, they also have to deal with the differences between dynamic languages. Whereas the refactoring tools for Java or C\# can be similar because the languages are very similar, for example Racket and Python are very different from each other.

Even having few refactoring operations, the available ones are very different from one language to another, however the rename, which is the most used refactoring operations, is available in every dynamic refactoring tool.

%\begin{table}[h]
%\label{tab-Comparing-All}
%\begin{tabular}{l|c|c|c|c}
%              & \multicolumn{1}{l|}{Dynamic?} & Refactoring Coverege & Interoperability & Major drawback				 		\\ \hline
%Griswold      & Yes                           & Medium               &                  &									    \\ \hline
%SmallTalk     & Yes                           & Medium               & No             	& Dependent of Unit tests \& maybe dead \\ \hline
%Javascript    & Yes                           & Very Low             &                  & Few Refactoring Operations 		    \\ \hline
%Bicycle       & Yes                           & Medium               & Yes              & Did not improve since 2004 			\\ \hline
%Rope          & Yes                           & Medium               & Yes              &                       			    \\ \hline
%Eclipse       & No                            & High                 & No               &                            			\\ \hline
%Visual Studio & No                            & Medium               & No               &                            			\\ \hline
%IntelliJ      & No                            & High                 & No               &                            			\\ \hline
%NetBeans      & No                            & Low                  & No               &                            			\\ \hline
%JBuilder      & No                            & Medium               & No               &                            			\\ \hline
%\end{tabular}
%\caption{Comparassion between Refactoring tools}
%\end{table}