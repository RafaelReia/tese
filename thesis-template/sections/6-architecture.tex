%!TEX root = ../report.tex

% 
% Architecture
% 

\section{Architecture}

%introduction

%explain why is important and stuff


%s-exp and def-uses-relationships

%An image will be awesome here
The architecture of the refactoring tool will consist in an AST and in the def-use-relations of the program.
The information gathered in the AST and in the def-use-relations with some preconditions is enough to ensure the correctness of the refactoring operations.
The AST in the Racket programing language is composed of s-expressions. Because a program is a list of s-expressions.
DrRacket provides the def-use-relations which is an important help to do the refactoring operations. The def-use-relations are visual represented as arrows in the DrRacket.


%use cases: To validate the architecture.


%introduce the validate cases xD
% rename improved
\subsection{Imported rename}
%this is a refactoring for racket
When using functions exported from another file or library it is necessary to require them, in other words to import.
After that the user can call the methods exported like they were defined in the file.
However those methods could have name collision or change it for a more adequate name. To do that the user would do a rename.

DrRacket already had implemented a rename, however the rename was not suit for renaming imported functions.
When renaming a imported function the original rename of DrRacket renamed the name all the functions imported from the same file and the name of the file. 
Making the program incorrect.


% add-prefix
\subsection{Add-prefix}
%this is a refactoring for racket
A programmer is using some functions from a library and then realizes that needs another library. when importing that library conflicts occur.
both libraries have function with the same name. The solution add a prefix for one of the requires.
That is really annoying because the user has to remember and change one by one all functions invocations.

The Add Prefix refactoring does all that for the user.

This is also useful when the name of the functions are similar, adding a prefix make it easier to distinguish between libraries.


% extract-function
\subsection{extract-function}

Extract functions is a simple and very useful refactoring as you can see in the use tables/overview. this is actually a good test to the architecture because it is largely used.
