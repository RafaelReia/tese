%!TEX root = ../report.tex

% 
% Evaluation
% 

\section{Evaluation}

%Explain how you are going to show your results (statistical data, cpu performance etc). Answer the following questions:
%\begin{itemize}
%  \item Why is this solution going to be better than others.
%  \item How am I going to defend that it is better.
%\end{itemize}

In order to evaluate the correctness of the refactoring operations the ideal would be to have a formal proof of the refactoring operations.
However, formal proofs are hard to do and take too much time that I do not have for this thesis.
The formal proofs that exist are usually done for theoretical languages and not for languages usually used.
Nevertheles, even professional refactoring tools such as Eclipse or NetBeans have errors, as proven in here. 

The solution to that problem is to prove informally. The prove consists in having a correct Racket program with unit tests, preferably developed with test-drive-developing, then apply some refactoring operations and then run the unit tests again. This allow to see that the refactoring operations are correct and do not introduce errors in the programs.

To evaluate the simplicity and usability of the refactoring operations it will be tested by having users following some use cases and test the users response to the tasks.