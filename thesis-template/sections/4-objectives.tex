%!TEX root = ../report.tex

% 
% Objectives
% 

\section{Objectives}

%The objectives of this thesis is to implement refactoring operations for unexperienced users.

%To do that it is important to have a language suited to beginners and a pedagogic IDE.

%Refactoring the program using refactoring tools is safer than manually, therefore it prevents introduction of bugs in a previous correct code.
%give example of rename, don not rename all the variables, there is some that are forgotten and stuff

%besides that when programming the unexperienced users tend to create a big function for everything and do not subdivide that function in smaller functions. (extract function refactoring)

The main goal of this thesis is to implement refactoring operations adequate with unexperienced users.
Because it is targeted for unexperienced users it is ideally designed for a language that is used to teach programming in introductory courses worldwide, for example Racket or Python.
It is also important to have a pedagogic environment instead of complex IDEs such as Eclipse.
Having a refactoring tool for unexperienced users is important because those users probably will not get the program right at first try.
Using a refactoring tool to do the refactoring operation will do a quicker and better job because it is safer than doing it manually.

Those operations must be:
\begin{itemize}
\item Correct
\item Simple to use
\item Useful
\end{itemize}

These characteristics are fundamental to create a refactoring tool targeted for unexperienced users.