%!TEX root = ../report.tex

% 
% Introduction
% 

\section{Introduction}


%Introduce the Refactoring Tools
Over time, software tends to change, new requirements appear or are adapted. 
Even after development new bugs are found or some critical new features are added.
%A piece of software that is used never stops changing and evolving.
These changes makes the artifact drift apart from its original design.
Typically these changes increase the complexity of the software, making it less readable and harder to change. 
Consequently making the quality lower and the maintenance costs higher.

These changes creates a need for continuously improving the software structure.

Refactoring is a meaning-preserving transformation or a set of meaning-preserving transformations that are meant to improve the program structure and the software quality \cite{bourquin2007high}.
Refactoring is a tedious and error prone activity, and because of that it is preferable to use a tool that provides automated support. %to the refactoring operations that the programmer intends to do. 
Therefore saving time and preventing the addition of errors to a previous correct program.

%
%talk about the lack of refactoring tools for unexperienced users and maybe the importance of using refactoring tools while learning how to program.
There is also a lack of tools designed for unexperienced users. 
Mainly because there is a lack of tools for dynamic languages and dynamic languages are the more adequate languages for learning.
An important refactoring operation an unexperienced user would do besides the rename is the extract-method. 
Unexperienced users tend to create one big function that does all the program work.
By having a refactoring tool that allows the user to extract functions the unexperienced users would improve the quality of the program in a safe way.

%give the example
% Example extract method. Unexperienced users create one big function that do all the program work. Create a function that should clearly have a function extracted from the inside.
An example of how a user would use the extract function is below.

\begin{lstlisting}[frame=single, caption=Before, language=Python]
def fibonacci(number=1):
	previous = 0
	current = 1
	history = []

	for i in range (0, number):
		aux = current + previous
		previous = current
		current = aux
		history.append(current)

	for i in range(number):
		print history[i]
\end{lstlisting}


\begin{lstlisting}[frame=single, caption=After, language=Python]
def fibonacci_history(number):
	previous = 0
	current = 1
	history = []
	for i in range (0, number):
		aux = current + previous
		previous = current
		current = aux
		history.append(current)
	return history

def print_history(history, number):
	for i in range(number):
		print history[i]

def fibonacci(number=1):
	history = fibonacci_history(number)
	print_history(history, number)
\end{lstlisting}


The user starts by choosing a set of expressions to extract, usually it is a  basic block or a set of basic blocks. 
%talk about refactoring for static languages and dynamic languages

The use of the refactoring tools is fully adopted by the object oriented and static programming languages with their IDEs (integrated development environment) support.
For example, languages such as Java with the IDEs, Eclipse\footnote{https://eclipse.org/}, IntelliJ\footnote{https://www.jetbrains.com/idea/} or NetBeans\footnote{https://netbeans.org/} and C\# with Visual Studio\footnote{https://www.visualstudio.com/}.
%When compared to the dynamic languages there is a lack of refactoring tools and refactoring operations.
%talk about the lack of refactoring tools for dynamic languages
The lack of information available during the refactoring about the program is the biggest difference between refactoring tools for static languages and refactoring tools for dynamic languages.
This difference is the main difficulty that made the refactoring tools for dynamic languages not evolve like the ones made for static languages. 
Therefore making the refactoring tools for static languages largely used and considered a common tool in contrast with the refactoring tools for dynamic languages.  %Does not sound good.

Despite that, the importance of dynamic languages is growing.
Mainly because they are becoming popular among unexperienced programmers, they are used a lot among the scientific community and there are new dynamic languages everyday. 
Dynamic languages are good for creating prototypes because it is easy and fast to create one. 
Finally they are easy to adopt and often used as a learning language, such as Scheme, Racket and Python. %% ADD REFS! use PEDRO RAMOS

%Conclude with the need of having such refactoring tool



%say what will be addressed in the next sections.

The Section 2 addresses the objectives for this thesis work. 
Section 3 describe some definitions related to refactoring tools, such as the types of refactoring tools.
Section 4 describes how the users refactoring, it compares refactoring tools for static languages, it describes refactoring tools for dynamic languages and languages independent refactoring tools.
Section 5 describes the architecture of the proposed solution. 
Section 6 explains how the tool will be evaluated and finally section 7 sums up the work.






%[REF]evolution survey do ments.
%\cite{drscheme} teste  \cite{drscheme_pegadogy} \cite{languages_scheme}
%Over time, software artifacts tends to change, in order to develop gradually, to expand %quando se esta a usar, depois de se usar durante algum tempo
%while being used and even during development, new requirements appear, the existing ones change, new bugs are found or some critical %SHINY! important
%new features are added.

 %[REF] Case study in refactoring functional programs.&& [REF] Refactoring: current research and future trends.


%Preserving the meaning is important because if the meaning changes, it transforms the program in a different program.

%[REF] FIND IT 
%The difference between Refactoring and restructuring is that Refactoring is used in literature to define the transformations that improve the program preserving the behavior in Object Oriented paradigm \cite{opdyke1992refactoring} \cite{fowlerrefactoring1999} whereas Restructuring is used for the rest. \cite{griswold1993automated} \cite{softrest1986} %[REF] 


%why we need refactoring tools[ref JunGL]






%The purpose of this paper is to show which refactoring tools exist for dynamic languages. %change this.




%The Section 2 addresses the objectives for this thesis work. Section 3 explore related work in refactoring and restructuring programs, some implemented restructuring tools and some implementations of language independent refactoring tools. Section 4 describes the architecture of the proposed solution. Section 5 explains how the tool will be evaluated and we conclude on section 6.



