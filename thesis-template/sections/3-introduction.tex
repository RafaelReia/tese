%!TEX root = ../report.tex

% 
% Introduction
% 

\section{Introduction}


%Introduce the Refactoring Tools
Over time, software tends to change, for instance when existing requirements are changed or new requirements are adopted. 
Even after development, bugs are found or some critical new features are added.
%A piece of software that is used never stops changing and evolving.
These changes make the software drift apart from its original design.
Typically these changes increase the complexity of the software, making it less readable and harder to change \cite{mens2003refactoring}. 
Consequently, it makes the quality lower and the maintenance costs higher.
Continuous changes create the need for continuously improving the software structure.

Refactoring is a meaning-preserving transformation or a set of meaning-preserving transformations that are meant to improve the program structure and therefore the software quality \cite{bourquin2007high}.
%However, because refactoring is a tedious and error prone activity, for example renaming one variable if the user forgets one it generates and error, it is preferable to use a too that provides automated support.
Thus, making refactoring a very desirable operation to maintain the software quality, but because it is a tedious and error prone activity, it is preferable to use a tool that provides automated support. 
Therefore saving time and preventing the addition of errors to a previously correct program.

%
%talk about the lack of refactoring tools for unexperienced users and maybe the importance of using refactoring tools while learning how to program.
Refactoring tools are important, however there is a lack of refactoring tools designed for unexperienced users.
They tend to create one big function that does all the program work and often follow the try-and-error paradigm to get the program correct.
This negatively affects the program quality, but fortunately, refactoring operations can improve the program structure.
By having refactoring operations available they could easily improve the program structure.
Because of the way unexperienced users usually program, by creating one big function, an important refactoring operation an unexperienced user would apply is the extract-function.
%Connecting phrase, if the unexperienced users decides to do that by hand, besides the tedious work that he has to do, he probably could forget stuff.
%By having a refactoring tool it would be quicker and safer.

By having a refactoring tool that allows the user to extract functions the unexperienced users would improve the quality of the program in a safe way.

%give the example
% Example extract method. Unexperienced users create one big function that do all the program work. Create a function that should clearly have a function extracted from the inside.
 Listing~\ref{lst:Fibonnacci} is an implementation of the Fibonacci function in Python and Listing~\ref{lst:FibonnacciRefactored} is an example of how a user would use the extract function refactoring operation.

\begin{lstlisting}[frame=single, caption=Fibonacci function first implementation, label={lst:Fibonnacci}, language=Python]
def fibonacci(number=1):
	previous = 0
	current = 1
	fib_numbers = []

	for i in range (0, number):
		aux = current + previous
		previous = current
		current = aux
		fib_numbers.append(current)

	for i in range(number):
		print fib_numbers[i]
\end{lstlisting}


\begin{lstlisting}[frame=single, caption=Fibonacci function after using extract function, label={lst:FibonnacciRefactored}, language=Python]
def fibonacci_seq(number):
	previous = 0
	current = 1
	fib_numbers = []
	for i in range (0, number):
		aux = current + previous
		previous = current
		current = aux
		fib_numbers.append(current)
	return fib_numbers

def print_fibonacci(fib_numbers, number):
	for i in range(number):
		print fib_numbers[i]

def fibonacci(number=1):
	fib_numbers = fibonacci_seq(number)
	print_fibonacci(fib_numbers, number)
\end{lstlisting}


In the extract function refactoring operation, the user starts by choosing a set of expressions to extract, usually it is a  basic block or a set of basic blocks.
A basic block is a portion of code that only has one entry point and only one exit point.
There are two functions extracted in the Listing~\ref{lst:FibonnacciRefactored}, the ``print\_fibonacci'' and ``'fibonacci\_seq''.
%talk about refactoring for static languages and dynamic languages

The use of the refactoring tools is fully adopted by the object oriented and static programming languages with their IDEs (integrated development environments) support.
For example, languages such as Java have IDEs like, Eclipse\footnote{https://eclipse.org/}, IntelliJ\footnote{https://www.jetbrains.com/idea/} or NetBeans\footnote{https://netbeans.org/} and C\# has Visual Studio\footnote{https://www.visualstudio.com/}.
%When compared to the dynamic languages there is a lack of refactoring tools and refactoring operations.
%talk about the lack of refactoring tools for dynamic languages
Unfortunately that is not the case for the refactoring tools for dynamic languages. 
The lack of information available during the refactoring about the program is the biggest difference between refactoring tools for static languages and refactoring tools for dynamic languages.
This difference is the main difficulty that made the refactoring tools for dynamic languages not evolve as the ones made for static languages. 
Therefore making the refactoring tools for static languages largely used and considered a common tool in contrast with the refactoring tools for dynamic languages.  %Does not sound good.

Nonetheless the importance of dynamic languages is growing. Dynamic languages are becoming popular among the scientific community.
In addition to that, dynamic languages are often used as a learning language in introductory courses around the world, for example, Scheme, Racket and Python.

%Conclude with the need of having such refactoring tool
A refactoring tool for a dynamic language adequate for unexperienced users would be an important step in filling the lack of refactoring tools for dynamic languages and it would also help the unexperienced users to have the first contact with refactoring tools and improve their code quality.


%say what will be addressed in the next sections.

The Section 2 addresses the objectives for this thesis work. 
Section 3 describes some definitions related to refactoring tools, such as the types of refactoring tools.
Section 4 describes how the users refactoring, it compares refactoring tools for static languages, it describes refactoring tools for dynamic languages, and language-independent refactoring tools.
Section 5 describes the architecture of the proposed solution. 
Section 6 explains how the tool will be evaluated and finally section 7 sums up the work.






%[REF]evolution survey do ments.
%\cite{drscheme} teste  \cite{drscheme_pegadogy} \cite{languages_scheme}
%Over time, software artifacts tends to change, in order to develop gradually, to expand %quando se esta a usar, depois de se usar durante algum tempo
%while being used and even during development, new requirements appear, the existing ones change, new bugs are found or some critical %SHINY! important
%new features are added.

 %[REF] Case study in refactoring functional programs.&& [REF] Refactoring: current research and future trends.


%Preserving the meaning is important because if the meaning changes, it transforms the program in a different program.

%[REF] FIND IT 
%The difference between Refactoring and restructuring is that Refactoring is used in literature to define the transformations that improve the program preserving the behavior in Object Oriented paradigm \cite{opdyke1992refactoring} \cite{fowlerrefactoring1999} whereas Restructuring is used for the rest. \cite{griswold1993automated} \cite{softrest1986} %[REF] 


%why we need refactoring tools[ref JunGL]






%The purpose of this paper is to show which refactoring tools exist for dynamic languages. %change this.




%The Section 2 addresses the objectives for this thesis work. Section 3 explore related work in refactoring and restructuring programs, some implemented restructuring tools and some implementations of language independent refactoring tools. Section 4 describes the architecture of the proposed solution. Section 5 explains how the tool will be evaluated and we conclude on section 6.



